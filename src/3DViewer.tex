\documentclass[12pt, a4paper]{article}
\usepackage{url}
\usepackage[utf8]{inputenc}
\usepackage{indentfirst}


\title{3DViewer v1.0}
\date{}

\begin{document}

\maketitle
\tableofcontents

\pagebreak

\section*{Description}

The 3DViewer program is designed to display 3D models parsed from files in format file.obj, in wireframe form.

It is able to use vertexes and facets only. Qt library is used for interface realization.

\section{Assembly}

To run the 3DViewer you will require following programs - Qt6 or later version ( including qmake), bash, GCC compiler,
GNU make, Mingw compiler (Qt6 contains it). For Windows OS it might be good to install Cygwin for \textbf{make tests}. Also for Windows it
might be necessary to add the PATH  of the System variables path, to avoid  possible errors. You need to add the paths
corresponding to the following to the PATH variable:

\noindent\textbf{C:\textbackslash{}..(path to Qt installed)\textbackslash{}Qt\textbackslash{}6.4.1(Qt version)\textbackslash{}mingw\_64(mingw version)\textbackslash{}bin}

\noindent\textbf{C:\textbackslash{}(path to Qt installed)\textbackslash{}Qt\textbackslash{}Tools\textbackslash{}mingw1120\_64(mingw tools version)\textbackslash{}bin}

May be it will be necessary to set equivalent paths in MacOS or Linux if assembly goes wrong.
 In order to build the program, use \textbf{make} in the root directory of the project.  Then you may execute program,
\textbf{3DViewer} is located in the folder \textbf{build/bin}.

\section{Installation }

Tested on MacOS  and Windows with Cygwin.
To install you may use \textbf{make install} , to install 3DViewer in your home directory.

\section{Testing}

Testing cover correctness of parsing of the .obj file and the Athenian transformations of the model.
For testing, you need to call \textbf{make tests}. Test coverage is called by \textbf{make gcov\_report}.


\section{Uninstall}

To uninstall the 3DViewer you should at  the root directory of the project and call \textbf{make uninstall}.
This will delete installation files and folders.


\vfill

\section{Main features of the 3DViewer}

In the 3DViewer program you are able to:
\begin{itemize}
    \item Upload and view model from the file in the fileName.obj
    \item Save the model as image in .bmp and .jpeg format
    \item Save GIF animation (640x480, 10fps, 5s)
    \item To see file path, amount of vertexes and facets
\end{itemize}

By using mouse, mouse wheel , spin boxes and slide bars:
\begin{itemize}
    \item Translating - move the model by a specified distance relative to the X, Y, Z axes
    \item Rotate the model by a given angle around its X, Y, Z axes
    \item Scale the model for a given value.
\end{itemize}

By using the Menu File:
\begin{itemize}
    \item Upload and view model from the file in the fileName.obj
    \item Close file
    \item Save the model as image in .bmp and .jpeg format
    \item Save GIF animation (640x480, 10fps, 5s)
\end{itemize}

By using the Menu Settings (pop-up dialog window appear):
\begin{itemize}
    \item Change widget background color
    \item Change facets parameters - colour, thickness and type of line (solid or dashed)
    \item Change vertices parameters - colour, size and type of points (none, circle or square)
    \item Change projection - parralle or central
    \item Restore default settings and start from the beginning!
\end{itemize}

The Program will save the settings if you close the model.

\end{document}
